\chapter{Introduction}
What seems to be a natural evolution for game playing experience is to bring the elimination of screens and devices in order to present the users with the possibility to physically interact with autonomous agents in their homes without the need to produce an entire virtual reality. This pretty new style of games has been recently defined as~\gls{pirg} and has as the main objective the exploitation of the real world (in both its dynamical unstructured and structured aspects) as environment and one or more real, physical, autonomous robots as a game opponents or companions~\cite{martinoia_physically_2013}.

Like commercial virtual games, the main aspect of~\gls{pirg}s is to produce a sense of entertainment and pleasure that can be ``consumed'' by a large number of users\footnote{In this work, since the player is an user for the gaming application both words ``player(s)'' and ``user(s)'' will be used interchangeably.}. Furthermore, an important aspect of autonomous robots and systems during the game should be, as commonly expected, an exhibition of rational behavior and, in this sense, they must be capable enough to play the role of opponents or teammates effectively, since by practical means people tend do not play with or against a dull entity~\cite{martinoia_physically_2013}.

Naturally, to help to come up with better agents its easy to think about extracting knowledge from co-existing agents in the sense of implementing some mechanism for modeling them, both to the extent of the quality of low-level actions they choose as well as interaction patterns with the system. This effort is justified in the context that being able to recognize other's intention may substantially improve one's capacity of taking better decision when acting upon the environment.  At least in the human's perspective, this ability is critical since interpersonal interactions presuppose the understanding of motivations, high-level plans, and estimation of future events~\cite{sukthankar_plan_2014}.

When it comes to extract useful information from agent's behavior, one can see at least two main different, yet related, approaches: \begin{inparaenum}[\itshape a\upshape)]\item Modeling for competitive advantage and \item Modeling for experience optimization\end{inparaenum}. In the former, techniques for evaluating pay-offs from interaction patterns, such that provided from \textit{game theory}, play an intense role not only to what concerns virtual interactions (with virtual agents) but real-world events involving humans as agents (e.g. trading, patrolling, competition) or physical robot entities. In order to get some competitive advantage against an adversary with private strategies and conflicting goals it is necessary to adapt to the dynamics of the environment caused by the game play~\cite{rofer_overview_2012}. In essence, this means that it is vital to pay attention to any information from the opponent behavior that might help to optimize the decision making process and find appropriated countermeasure actions.

The focus on modeling behavior for experience optimization, is much related to the idea of extracting useful features from a user in order to adjust parameters that are correlated with his experience in a platform or product, all for the sake of offering a better product or helping the user to achieve some particular goals. In a gaming scenario, in turn, this notion is commonly applied when designers attempt to define a mechanism capable of adjusting the difficulty or general appearance of the game all in the expectation of rising the player entertainment. Very traditionally the sense of game difficulty is designed to increase along the course of the experience, which can either happen in a linear fashion or through steps represented by the levels or phases, where a player is forced to select the difficulty level through a set of discrete option (easy, medium, hard, very hard). However, given the observation that very often this ``static'' way of setting up a difficulty curve is not accurate enough and it may not account for the difference between players or even the different rates of leaning of each of them, in principle, it turns out useful and natural to think about coming up with modeling techniques that may help to empower the player experience. 

Such models, however, are greatly impacted by the type of scenario they are applied to. Normally, in a game scenario a computer-controlled agent may receive a noise-free sensory data which is obviously not possible to occur in a real-world scenario, specially in robotic applications such of that of~\gls{pirg}. The general suspicion about a full spread of such solutions, mainly in virtual game development, is that they ultimately takes the control away from the design and put it basically in the code, which has obvious drawbacks, ranging from high-demand for computational resources and storage to general game behavior~\cite{hunicke_ai_2004}. 

In summary, it would be important to have a~\gls{pirg} autonomous robot that could be able to automatically adjust its behavior such that it may be likely to match the user's skill and by doing so maintain the user engaged. Also, it is important to, by aiming those objectives, make the robots appear more intelligent. 

In this thesis work, after providing a panorama of approaches that take inspiration from \textit{artificial intelligence} (AI) and \textit{machine learning} (ML) techniques in order to tackle the problem of modeling co-existing agent's behavior and activity (including humans) in games and robots, we propose models and report of results showing effort in addressing the design of better agents for~\gls{pirg}. 

The scope of this document is heavily centered on model proposals that can latently model player activities and general behavior.  Next section details our objects and research organization.

\section{Research questions, hypothesis and objectives}	
Since, in any kind of~\gls{pirg}, autonomous robot are supposed to be perceived as smart entities, the key point for investigation deals with finding good answers for two main questions:

\begin{itemize}
\item How to discover players types and quantify player behavior in~\gls{pirg}.
\item To what extent does adaptation impact reported fun?
\end{itemize}

From this, the objective focus on the exploitation of~\gls{ml} techniques to the design of better~\gls{pirg} robotic agents, which should lead to more engaging playmates, possibly capable of performing behavior/strategy adjustment. The hypothesis is that~\gls{ml} would help to decrease predictability in robot behavior and introduce game dynamics capable of considerably empower the user’s engagement, making so that the agents can be well accepted as game companions. Specifically, the general goals were:

\begin{itemize}
\item To perceive and report the impact caused by the use of~\gls{ml}-based techniques in the design of~\gls{pirg} from the view of human-robot interaction.
\item To study ways of implementing user's behavior modeling in~\gls{pirg} robots by reasoning about data coming from player tracking.
\item To explore ways of strategy adjustment using information about past interactions and experiences (player's typical behaviors, preferences, etc.).
\end{itemize}

It is supposed that autonomous and learning systems that encompass perception, action and communication in a unified and principled way via~\gls{ml}-based techniques lay at the core of a new frontier for robotics, and~\gls{pirg}s in particular. The objectives also aim to keep control on technical constraints by exploring the use of cheap sensors, and algorithms requiring little power ("green algorithms") to be executed in real time and non-structured environments, as a way to enable the spread of~\gls{pirg} in the society, making them reach the market in large scale.

\section{Thesis outline}
The thesis is divided into the following chapters.

\begin{itemize}
\item\emph{Chapter~\ref{ch:art}:} Many approaches for tailoring game experience to players had been proposed over the years. The chapter presents an overview of such literature, placing~\gls{pirg} as a relatively new area of research.
\item\emph{Chapter~\ref{ch:foundation}:} The game environment and robot platform are at the core of a~\gls{pirg} application. In this chapter we detail our selected environment and adopted robotic platform. Other mathematical background are provided.
\item\emph{Chapter~\ref{ch:activity}:} One step towards allowing effective player modeling is the implementation of an activity recognition system. In this chapter we describe the efforts on exploiting a simple input data transformation for the recognition of motion primitives in acceleration pattern akin to archetypal activities in the game scenario.
\item\emph{Chapter~\ref{ch:modeling}:} Player behavior modeling is the backbone of adaptive behavior strategy for playing robots. The chapter presents new idea for latent player behavior modeling.
\item\emph{Chapter~\ref{ch:adaptation}:} Adaption is by no means a trivial task specially in the context of~\gls{pirg}. In this chapter we detail a system to actively selected appropriate difficulty settings for the human player.
\item\emph{Chapter~\ref{ch:deception}:} Engagement is believed to be related to several factors and one of such is the level of information about the opponent actions. In this chapter we present a study case of the use of deceptive motion during play. Results indicate that on our particular game scenario the human capacity to correctly identify the robot deceptive intentions is blurred possibly by the intensive cognitive load demanded by the activity.
\item\emph{Chapter~\ref{ch:key}:} The design of~\gls{pirg} is an intensive process. In this chapter we provide some key-issues regarding how one may develop successful engaging experiences.
\item\emph{Chapter~\ref{ch:future}:} Concludes the thesis and details further direction for our research.
\end{itemize}

\section{Paper contributions}
\begin{itemize}
\item \fullcite{oliveira_learning_2018}
\item \fullcite{oliveira_modeling_2017}
\item \fullcite{oliveira_activity_2017}
\end{itemize}
